\href{https://opensource.org/licenses/MIT}{\tt } \section*{R\+OS Beginner Tutorials}

A beginner R\+OS package that implements simple publisher-\/subscriber nodes in R\+OS

\subsection*{Installations}

To install R\+OS refer \href{http://wiki.ros.org/ROS/Installation}{\tt this link}.

Preferred Ubuntu 18.\+04, R\+OS Melodic

\#\#\# Step 1. Clone repository and build the package 
\begin{DoxyCode}
git clone --recursive https://github.com/kavyadevd/beginner\_tutorials.git
cp <repository\_path> <catkin\_workspace\_path/src/>
cd <catkin\_workspace>
source ./devel/setup.bash
catkin\_make
\end{DoxyCode}


\#\#\# Step 2. Make sure that a roscore is up and running 
\begin{DoxyCode}
roscore
\end{DoxyCode}
 \#\#\# Step 3. In new terminal run publisher node 
\begin{DoxyCode}
rosrun beginner\_tutorials talker
\end{DoxyCode}
 \#\#\# Step 4. In new terminal run subscriber node 
\begin{DoxyCode}
rosrun beginner\_tutorials listener
\end{DoxyCode}


\paragraph*{To run launch file}


\begin{DoxyCode}
roslaunch beginner\_tutorials LaunchTalkerListener.launch frequency:=<desired-frequency>
\end{DoxyCode}
 This command will launch published and subscriber.

\#\#\#\# To run service 
\begin{DoxyCode}
rosservice call /ServiceFile "<desired-custom-message>"
\end{DoxyCode}


\#\#\#\# To invoke rqt logger and console on separate terminals run the following two commands 
\begin{DoxyCode}
rosrun rqt\_console rqt\_console
rosrun rqt\_logger\_level rqt\_logger\_level
\end{DoxyCode}


\subsection*{Plugins}


\begin{DoxyItemize}
\item Cpp\+Check\+Eclipse

To install and run cppcheck in Eclipse
\begin{DoxyEnumerate}
\item In Eclipse, go to Window -\/$>$ Preferences -\/$>$ C/\+C++ -\/$>$ cppcheclipse. Set cppcheck binary path to \char`\"{}/usr/bin/cppcheck\char`\"{}.
\item To run C\+P\+P\+Check on a project, right-\/click on the project name in the Project Explorer and choose cppcheck -\/$>$ Run cppcheck.
\item To run on terminal ```bash cppcheck --enable=all --std=c++11 -\/I include/ --suppress=missing\+Include\+System \$( find . -\/name $\ast$.cpp -\/or -\/name $\ast$.h $\vert$ grep -\/vE -\/e \char`\"{}$^\wedge$./build/\char`\"{} -\/e \char`\"{}$^\wedge$./vendor/\char`\"{}) $>$ Results/cppcheckoutput.\+xml ``` Results are present at Results/cppcheckoutput.\+xml
\end{DoxyEnumerate}
\item Cpplint
\begin{DoxyEnumerate}
\item To run cpplint on terminal 
\begin{DoxyCode}
cpplint $( find . -name *.cpp | grep -vE -e "^./build/" -e "^./vendor/") $( find . -name *.hpp | grep -vE
       -e "^./build/" -e "^./vendor/") >                    Results/cpplintoutput.txt
\end{DoxyCode}
 Results are present at Results/cpplintoutput.\+xml
\end{DoxyEnumerate}
\item Google C++ Style

To include and use Google C++ Style formatter in Eclipse
\begin{DoxyEnumerate}
\item In Eclipse, go to Window -\/$>$ Preferences -\/$>$ C/\+C++ -\/$>$ Code Style -\/$>$ Formatter. Import \href{https://raw.githubusercontent.com/google/styleguide/gh-pages/eclipse-cpp-google-style.xml}{\tt eclipse-\/cpp-\/google-\/style} and apply.
\item To use Google C++ style formatter, right-\/click on the source code or folder in Project Explorer and choose Source -\/$>$ Format
\end{DoxyEnumerate}
\item Doxygen

The H\+T\+ML page for project outlines can be generated using the following commands
\begin{DoxyEnumerate}
\item doxygen -\/g
\item doxygen Doxyfile
\end{DoxyEnumerate}
\end{DoxyItemize}

\subsection*{Licensing}

The project is licensed under \href{https://opensource.org/licenses/MIT}{\tt M\+IT License}. Click \href{https://github.com/kavyadevd/beginner_tutorials/blob/main/LICENSE}{\tt here} to know more 